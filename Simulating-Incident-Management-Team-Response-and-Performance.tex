\documentclass[fancy, oneside, mastersfancy, ms]{byuthesis}
\usepackage{bookmark}


\usepackage{color}
\usepackage{fancyvrb}
\newcommand{\VerbBar}{|}
\newcommand{\VERB}{\Verb[commandchars=\\\{\}]}
\DefineVerbatimEnvironment{Highlighting}{Verbatim}{commandchars=\\\{\}}
% Add ',fontsize=\small' for more characters per line
\usepackage{framed}
\definecolor{shadecolor}{RGB}{241,243,245}
\newenvironment{Shaded}{\begin{snugshade}}{\end{snugshade}}
\newcommand{\AlertTok}[1]{\textcolor[rgb]{0.68,0.00,0.00}{#1}}
\newcommand{\AnnotationTok}[1]{\textcolor[rgb]{0.37,0.37,0.37}{#1}}
\newcommand{\AttributeTok}[1]{\textcolor[rgb]{0.40,0.45,0.13}{#1}}
\newcommand{\BaseNTok}[1]{\textcolor[rgb]{0.68,0.00,0.00}{#1}}
\newcommand{\BuiltInTok}[1]{\textcolor[rgb]{0.00,0.23,0.31}{#1}}
\newcommand{\CharTok}[1]{\textcolor[rgb]{0.13,0.47,0.30}{#1}}
\newcommand{\CommentTok}[1]{\textcolor[rgb]{0.37,0.37,0.37}{#1}}
\newcommand{\CommentVarTok}[1]{\textcolor[rgb]{0.37,0.37,0.37}{\textit{#1}}}
\newcommand{\ConstantTok}[1]{\textcolor[rgb]{0.56,0.35,0.01}{#1}}
\newcommand{\ControlFlowTok}[1]{\textcolor[rgb]{0.00,0.23,0.31}{#1}}
\newcommand{\DataTypeTok}[1]{\textcolor[rgb]{0.68,0.00,0.00}{#1}}
\newcommand{\DecValTok}[1]{\textcolor[rgb]{0.68,0.00,0.00}{#1}}
\newcommand{\DocumentationTok}[1]{\textcolor[rgb]{0.37,0.37,0.37}{\textit{#1}}}
\newcommand{\ErrorTok}[1]{\textcolor[rgb]{0.68,0.00,0.00}{#1}}
\newcommand{\ExtensionTok}[1]{\textcolor[rgb]{0.00,0.23,0.31}{#1}}
\newcommand{\FloatTok}[1]{\textcolor[rgb]{0.68,0.00,0.00}{#1}}
\newcommand{\FunctionTok}[1]{\textcolor[rgb]{0.28,0.35,0.67}{#1}}
\newcommand{\ImportTok}[1]{\textcolor[rgb]{0.00,0.46,0.62}{#1}}
\newcommand{\InformationTok}[1]{\textcolor[rgb]{0.37,0.37,0.37}{#1}}
\newcommand{\KeywordTok}[1]{\textcolor[rgb]{0.00,0.23,0.31}{#1}}
\newcommand{\NormalTok}[1]{\textcolor[rgb]{0.00,0.23,0.31}{#1}}
\newcommand{\OperatorTok}[1]{\textcolor[rgb]{0.37,0.37,0.37}{#1}}
\newcommand{\OtherTok}[1]{\textcolor[rgb]{0.00,0.23,0.31}{#1}}
\newcommand{\PreprocessorTok}[1]{\textcolor[rgb]{0.68,0.00,0.00}{#1}}
\newcommand{\RegionMarkerTok}[1]{\textcolor[rgb]{0.00,0.23,0.31}{#1}}
\newcommand{\SpecialCharTok}[1]{\textcolor[rgb]{0.37,0.37,0.37}{#1}}
\newcommand{\SpecialStringTok}[1]{\textcolor[rgb]{0.13,0.47,0.30}{#1}}
\newcommand{\StringTok}[1]{\textcolor[rgb]{0.13,0.47,0.30}{#1}}
\newcommand{\VariableTok}[1]{\textcolor[rgb]{0.07,0.07,0.07}{#1}}
\newcommand{\VerbatimStringTok}[1]{\textcolor[rgb]{0.13,0.47,0.30}{#1}}
\newcommand{\WarningTok}[1]{\textcolor[rgb]{0.37,0.37,0.37}{\textit{#1}}}

\providecommand{\tightlist}{%
  \setlength{\itemsep}{0pt}\setlength{\parskip}{0pt}}\usepackage{longtable,booktabs,array}
\usepackage{calc} % for calculating minipage widths
% Correct order of tables after \paragraph or \subparagraph
\usepackage{etoolbox}
\makeatletter
\patchcmd\longtable{\par}{\if@noskipsec\mbox{}\fi\par}{}{}
\makeatother
% Allow footnotes in longtable head/foot
\IfFileExists{footnotehyper.sty}{\usepackage{footnotehyper}}{\usepackage{footnote}}
\makesavenoteenv{longtable}
\usepackage{graphicx}
\makeatletter
\def\maxwidth{\ifdim\Gin@nat@width>\linewidth\linewidth\else\Gin@nat@width\fi}
\def\maxheight{\ifdim\Gin@nat@height>\textheight\textheight\else\Gin@nat@height\fi}
\makeatother
% Scale images if necessary, so that they will not overflow the page
% margins by default, and it is still possible to overwrite the defaults
% using explicit options in \includegraphics[width, height, ...]{}
\setkeys{Gin}{width=\maxwidth,height=\maxheight,keepaspectratio}
% Set default figure placement to htbp
\makeatletter
\def\fps@figure{htbp}
\makeatother
\newlength{\cslhangindent}
\setlength{\cslhangindent}{1.5em}
\newlength{\csllabelwidth}
\setlength{\csllabelwidth}{3em}
\newlength{\cslentryspacingunit} % times entry-spacing
\setlength{\cslentryspacingunit}{\parskip}
\newenvironment{CSLReferences}[2] % #1 hanging-ident, #2 entry spacing
 {% don't indent paragraphs
  \setlength{\parindent}{0pt}
  % turn on hanging indent if param 1 is 1
  \ifodd #1
  \let\oldpar\par
  \def\par{\hangindent=\cslhangindent\oldpar}
  \fi
  % set entry spacing
  \setlength{\parskip}{#2\cslentryspacingunit}
 }%
 {}
\usepackage{calc}
\newcommand{\CSLBlock}[1]{#1\hfill\break}
\newcommand{\CSLLeftMargin}[1]{\parbox[t]{\csllabelwidth}{#1}}
\newcommand{\CSLRightInline}[1]{\parbox[t]{\linewidth - \csllabelwidth}{#1}\break}
\newcommand{\CSLIndent}[1]{\hspace{\cslhangindent}#1}

\usepackage{booktabs}
\usepackage{longtable}
\usepackage{array}
\usepackage{multirow}
\usepackage{wrapfig}
\usepackage{float}
\usepackage{colortbl}
\usepackage{pdflscape}
\usepackage{tabu}
\usepackage{threeparttable}
\usepackage{threeparttablex}
\usepackage[normalem]{ulem}
\usepackage{makecell}
\usepackage{xcolor}
\usepackage{siunitx}
\usepackage{booktabs}
\usepackage{longtable}
\usepackage{array}
\usepackage{multirow}
\usepackage{wrapfig}
\usepackage{float}
\usepackage{colortbl}
\usepackage{pdflscape}
\usepackage{tabu}
\usepackage{threeparttable}
\usepackage{threeparttablex}
\usepackage[normalem]{ulem}
\usepackage[utf8]{inputenc}
\usepackage{makecell}
\usepackage{xcolor}
\makeatletter
\makeatother
\makeatletter
\@ifpackageloaded{bookmark}{}{\usepackage{bookmark}}
\makeatother
\makeatletter
\@ifpackageloaded{caption}{}{\usepackage{caption}}
\AtBeginDocument{%
\ifdefined\contentsname
  \renewcommand*\contentsname{Table of contents}
\else
  \newcommand\contentsname{Table of contents}
\fi
\ifdefined\listfigurename
  \renewcommand*\listfigurename{List of Figures}
\else
  \newcommand\listfigurename{List of Figures}
\fi
\ifdefined\listtablename
  \renewcommand*\listtablename{List of Tables}
\else
  \newcommand\listtablename{List of Tables}
\fi
\ifdefined\figurename
  \renewcommand*\figurename{Figure}
\else
  \newcommand\figurename{Figure}
\fi
\ifdefined\tablename
  \renewcommand*\tablename{Table}
\else
  \newcommand\tablename{Table}
\fi
}
\@ifpackageloaded{float}{}{\usepackage{float}}
\floatstyle{ruled}
\@ifundefined{c@chapter}{\newfloat{codelisting}{h}{lop}}{\newfloat{codelisting}{h}{lop}[chapter]}
\floatname{codelisting}{Listing}
\newcommand*\listoflistings{\listof{codelisting}{List of Listings}}
\makeatother
\makeatletter
\@ifpackageloaded{caption}{}{\usepackage{caption}}
\@ifpackageloaded{subcaption}{}{\usepackage{subcaption}}
\makeatother
\makeatletter
\@ifpackageloaded{tcolorbox}{}{\usepackage[skins,breakable]{tcolorbox}}
\makeatother
\makeatletter
\@ifundefined{shadecolor}{\definecolor{shadecolor}{rgb}{.97, .97, .97}}
\makeatother
\makeatletter
\makeatother
\makeatletter
\makeatother

\title{Simulating Incident Management Team Response and Performance}
\author{Daniel Jarvis}

% On the custom title page, use the same title, but format as you like
\customtitle{Simulating Incident Management Team Response and
Performance}

% This is the date of graduation
\date{2023-06-25}

% If your degree is not a PhD or MS, then you can overwrite the degree using 
% the \degree command: \degree{Bachelors of Basics}

% Your department
\department{Civil and Construction Engineering}

% The names of your committee members
\committeechair{Gregory S. Macfarlane}
  \committeemember{Grant G. Schultz}
  \committeemember{Gustavious P. Williams}

\keywords{
    Demand Responsive Transport; Incident Management Teams; MATSim --
Multi-Agent Transport Simulation; Network Change Events;  
    Transportation Modeling
}

\begin{document}

\frontmatter
\titlepage
\cleardoublepage

\customtitlepage
\cleardoublepage


  \begin{abstract}
The abstract is a crucial component of any scientific paper, as it
provides a summary of the research and its main findings. This paper
provides guidelines for writing an effective scientific abstract. The
first step is to identify the key elements of the research, such as the
research question, methods, results, and conclusions. Next, the abstract
should be written in a clear and concise manner, using simple language
and avoiding technical jargon. The abstract should also be structured,
with a clear introduction, methods section, results section, and
conclusion. Additionally, the abstract should accurately and succinctly
convey the main findings of the research, highlighting the significance
and implications of the work. By following these guidelines, researchers
can ensure that their abstract effectively communicates the key aspects
of their research and attracts the attention of potential readers. -
Written by ChatGPT
\end{abstract}
\cleardoublepage

\begin{acknowledgments}
I would like to express my sincere gratitude to the Utah Department of
Transportation for their generous financial support, which was
instrumental in bringing this research and thesis project to fruition.
My heartfelt thanks also go to the technical advisory committee, whose
guidance and constructive feedback have been invaluable throughout the
course of this research. Their deep insights into Utah's traffic
incident management programs played a crucial role in shaping this work.

I am also immensely grateful to my fellow research assistants at BYU,
Joel Hyer, Harrison Holdsworth, and Brynn Woolley. Joel, through his
work on the IMT Performance Phase III project, provided essential
information on IMT performance, as well as incident data and metrics,
all of which were pivotal to this project. Harrison's contributions were
multifaceted; he not only compiled research on IMT optimization, but
also assisted in writing the literature review for this report and
served as a vital link between Joel's project and my own research.
Brynn's expertise was invaluable in the development of the
transportation model used in this project, and her assistance in
analyzing its outputs, as well as her help in the production and editing
of this thesis document, were greatly appreciated.

I want to express my profound appreciation to my faculty advisors and
the members of the thesis committee, Drs. Macfarlane, Schultz, and
Williams. Special gratitude is extended to Dr.~Macfarlane, whose role as
an outstanding faculty advisor, teacher, mentor, and friend has been
invaluable throughout my academic journey. His patience, counsel, and
support have been crucial at every phase of this project. I am deeply
thankful for his belief in my potential and his consistent
encouragement, which have been instrumental in my personal growth and
the success of this research.
\end{acknowledgments}
\cleardoublepage

	\tableofcontents*
	\cleardoublepage

	\listoffigures
	\cleardoublepage

	\listoftables
	\cleardoublepage

\mainmatter
\bookmarksetup{startatroot}

\hypertarget{introduction}{%
\chapter{Introduction}\label{introduction}}

Incident Management Teams (IMT) are service vehicles that collaborate
with highway patrol teams to manage traffic after an incident and
provide timely roadside assistance. They are strategically important for
improving highway operations, particularly during peak traffic times,
helping to effectively alleviate congestion and associated user costs.
Capable of quickly addressing a range of incidents, from minor vehicle
breakdowns to severe multi-car collisions, IMT are crucial in
controlling traffic and restoring normal flow on the roadways.

States like Utah have long benefited from IMT, witnessing notable
reductions in congestion and traffic-related costs (Bennett, 2021). The
effectiveness of IMT, influenced by factors such as response times
(Schultz et al., 2019), fleet size (Kim et al., 2012), and deployment
locations (Ozbay et al., 2013), is well-documented. However, the current
understanding primarily stems from ad-hoc models and independent
initiatives, leaving a gap in regional-scale traffic delay modeling
associated with incident management.

A study conducted by Kaddoura \& Nagel (2018) highlights the potential
of large-scale traffic models to evaluate the regional impact of
incidents. However, their model does not evaluate the impact of IMT
strategies. This research aims to address this limitation, merging
insights on IMT effectiveness with a large-scale regional traffic model.
This approach is meant to highlight the impacts of traffic incidents and
IMT interventions on a simulated network.

In this research we utilized the the Mutli-Agent Transportion Simulation
(MATSim) to model roadway conditions along the Wasatch Front region of
Utah. The model was used to evaluate IMT effectiveness within various
traffic incident scenarios. After conducting 60 simulations, we compared
their outcomes and evaluated the effectiveness of IMT at improving
traffic conditions and reducing congestion.

This paper contributes to the existing body of knowledge on IMT
effectiveness and traffic incident modeling. It underscores the
importance of strategic IMT deployment and the necessity for continuous
evaluation and adaption of IMT strategies to meet the evolving demands
of highway operation systems. The paper proceeds in a typical order.
\protect\hyperlink{sec-literature}{Literature Review} contains a
discussion of previous research into IMT effectiveness and optimization.
\protect\hyperlink{sec-methods}{Methodology} describes the simulation
and scenario construction, while
\protect\hyperlink{sec-results}{Results} presents the findings of the
analysis alongside a discussion of their implications. The paper
concludes in \protect\hyperlink{sec-conclusions}{Conclusions} with an
outline of future research motivated by this study's limitations.

\bookmarksetup{startatroot}

\hypertarget{sec-literature}{%
\chapter{Literature Review}\label{sec-literature}}

Traffic incident management in general---and IMT in particular---are not
strictly new innovations. The Federal Highway Administration (FHWA)
publishes the \emph{Traffic Incident Management Handbook} (FHWA, 2000),
which defines traffic incident management as:

\begin{quote}
\emph{The systematic, planned, and coordinated use of human,
institutional, mechanical, and technical resources to reduce the
duration and impact of incidents and improve the safety of motorists,
crash victims, and incident responders. (p.~1-1)}
\end{quote}

The handbook details the process of how to implement a traffic incident
management program as well as improve it. The manual covers various
aspects of incident management, including the responsibilities of
emergency medical teams, law enforcement, and other responding entities.
For this research, we focus on the dedicated traffic incident management
teams operated by departments of transportation or similar agencies and
not on other types of first responders.

FHWA has established performance measures to develop a framework to
quantify improvements to IMT operations (FHWA, 2000). A specific measure
related to this research is roadway clearance time (RCT): the time
between the first recordable awareness of the incident and the time all
lanes open for traffic flow. Numerous studies have assessed the impact
of Traffic Incident Management programs on traffic conditions, utilizing
the performance measures provided by FHWA. A particularly noteworthy
study conducted by Schultz et al. (2019) explores the relationship
between IMT response time (RT) and RCT. This research leveraged
interconnected data from the Utah Department of Transportation (UDOT)
and the Utah Highway Patrol (UHP), aiming to quantify the traffic
improvements resulting from swift IMT interventions at incident sites.
Analyzing 121 incidents, the study found that a one-minute delay in IMT
response correlated with a 0.8-minute increase in RCT. This delay also
impacted an additional 93 vehicles, added roughly 34.6 minutes to the
network's total estimated travel time, and resulted in an extra \$925 in
excess user costs (ECU). Schultz et al. (2019) established a clear
connection between timely IMT responses and improved traffic conditions,
underscoring the importance of rapid intervention.

Skabardonis et al. (1998) confirmed the effectiveness of IMT in their
study, concluding that IMT in California effectively reduced incident RT
and ECU. Skabardonis et al. (1998) found that, on average, total
incident RT was 15 minutes longer when California Highway Patrol (CHP)
responded without the support of IMT. Using a system to assign a cost
per traveler per unit of time to vehicles in the observed area, the
authors determined that IMT units had a cost-to-benefit ratio of 5:1.
They also concluded that CHP officers spent less time on incidents
(including vehicle breakdowns) when assisted by IMT services.

\hypertarget{sec-lit_imt_opt}{%
\section{IMT Optimization}\label{sec-lit_imt_opt}}

Given the evidence that IMT programs improve traffic conditions and
reduce costs for government entities and individuals, it becomes crucial
to further research avenues to maximize these benefits. One possible
strategy is the strategic placement of IMT units, optimizing their
spatial effectiveness to enhance their impact. Enhancing IMT programs
often focuses on the precise deployment of individual trucks and the
strategic positioning of IMT depots---locations where inactive trucks
await dispatch. For scenarios where IMT vehicles are actively on patrol,
research often concerns designing an efficient service area. Various
methodologies have been applied to tackle this allocation challenge.
While some studies employ statistical models, incorporating a range of
variables to maximize specific performance measures given constraints,
others opt for digital modeling as a solution.

For instance, Ozbay et al. (2013) designed a mixed-integer programming
model with probabilistic constraints to optimize the allocation of IMT
across ``depots'' or staging areas in New Jersey. This innovative
approach, grounded in known probabilities of various incident types,
strategically positions IMT units to respond to incidents, taking into
account future probabilities on the network. The primary goals were to
minimize incident management costs and maximize the likelihood that
every incident receive assistance. The model was applied to a simplified
South Jersey Highway network, utilizing traffic incident data from the
region to inform demand distribution. Through this application, an
optimal number of depots and truck assignments were determined. However,
the lack of a comparative analysis with pre-existing depot and unit
distributions meant that the exact improvements yielded by the model
remained unquantified.

Where Ozbay et al. (2013) focused on optimizing the IMT allocation in
specific zones, others have researched their effectiveness as roaming
entities. Lou et al. (2011) developed a mixed-integer nonlinear
optimization model and proposed different algorithms to minimize the RT
of IMT. They modeled IMT within specific freeway sections, and incident
frequencies were generated randomly on the network, given the mean and
standard deviations of incident occurrence on each link in the network.
The study focused on developing and optimizing these algorithms for
broad implementation rather than focusing on any particular network or
reducing response times in specific areas. They implemented a template
Sioux Falls network into the model as a practical demonstration.
Compared to the existing deployment plan in Sioux Falls, the
algorithm-generated plans could potentially reduce total RT by
16.5-20.8\%.

Each of the studies mentioned attempts to understand optimal IMT
deployment based on ad-hoc models, specially constructed utility
functions, or similar stand-alone efforts. While their results provide
valuable insights, they might be limited in their scope, as they do not
explicitly attempt to model the traffic delay associated with incident
management at a large scale. Research modeling the effects of incidents
on region-scale traffic networks is a recent innovation, providing a
more holistic view of the impact. This approach is potentially
beneficial for comprehensively assessing the effectiveness of IMT and
paves the way for our subsequent discussion on Incident Modeling, where
we will discuss the advancements and applications of this innovative
research domain.

\hypertarget{sec-lit_inc_mod}{%
\section{Incident Modeling}\label{sec-lit_inc_mod}}

The majority of traffic models utilized to study the impacts of
incidents and incident management are Dynamic Traffic Assignment (DTA)
models. These models demonstrate how congestion and travel times
fluctuate over time, varying for different vehicles (Boyles, 2018).
Their capability to represent variance over time makes DTA models
particularly advantageous for depicting the unpredictability of
incidents. A study conducted by Sisiopiku et al. (2007), which explored
the effects of congestion, underscored the applications of
simulation-based DTA modeling in incident management. It advocated for
dynamic assignment as the optimal approach for incident modeling.
Sisiopiku describes their methodology as follows:

\begin{quote}
\emph{The overall approach in this study is to use the DTA capabilities
to support decision-making for incident management. DTA is particularly
appropriate for studying short-term planning applications such as
evaluating various incident management options (p.~111).}
\end{quote}

In their study, Sisiopiku employed a DTA model to understand the impacts
of diverse incident scenarios and to assess the effectiveness of
potential incident management strategies and traffic control methods.
The research commenced with a baseline scenario under standard
conditions, establishing a reference point for evaluating incident
impacts. Subsequent scenarios simulated incidents without notifying
drivers, with variations in the duration and severity of the incidents.
The final scenario replicated the previous one but incorporated
information provision to drivers, enabling them to optimize their routes
and access pre-planned diversion paths, with guidance from Variable
Message Signs. Conducted in Birmingham and Chicago, the study
illustrated that post-incident information provision could result in
travel time savings and reduced traffic delays, highlighting the DTA
model's utility in simulating the effects of incidents and evaluating
traffic management and control strategies.

The study utilized the Visual Interactive System for Transport
Algorithms (VISTA) as the specific modeling tool. Despite its efficacy,
VISTA was critiqued by Wirtz et al. (2005) for its inherent assumption
that all drivers have perfect travel time information for routing to the
optimal path. For example, Sisiopiku's study assumed a 100\% compliance
rate with the provided diversion routes in their model. However, Wirtz's
research revealed that ``less-informed drivers spend more time traveling
than necessary, representing a departure from the user-optimal traffic
conditions simulated by VISTA.'' This critique does not invalidate
Sisiopiku's findings but highlights the limitations of VISTA in
reflecting real-world travel behavior.

VISTA is categorized as a large-scale, or mesoscopic, model suitable for
modeling extensive networks. Conversely, small-scale, or microscopic,
models, capable of tracking precise vehicle locations, driver behavior,
and vehicle characteristics, offer a highly realistic representation but
are impractical for large regions (Boyles, 2018). In Australia, Dia \&
Cottman (2006) utilized the Visual Simulation Model (VISSIM) to assess
the impacts of incident management on two arterial routes connecting the
western suburbs of Brisbane to the Central Business District. Although
VISSIM is a primary traffic modeling tool for UDOT, its detailed
precision renders it impractical for modeling incidents and IMT impacts
along the Utah Wasatch Front. Given this research's scope, a large-scale
dynamic model would likely be the most suitable choice.

Interestingly, in 2002, Pal \& Sinha (2002) developed a model to
replicate the impacts of incidents and IMT on traffic conditions in
Indiana, with objectives similar to this study, utilizing overall
traffic conditions as the performance indicator for IMT effectiveness.
Various configurations of response vehicles were simulated,
incorporating probability distributions of crash data, vehicle speed,
and roadway carrying capacity. The study's results informed
recommendations on fleet size, operation hours, patrol area design, and
dispatching policy improvements. However, since mesoscopic traffic
simulations in 2002 couldn't simulate incident response units, Pal \&
Sinha had to create their model from scratch. Unfortunately, their model
and methodology seem tailored specifically for their study and may not
be applicable to this project's context. Additionally, it's worth noting
that the region their model covered is smaller than the area served by
the Utah IMT program.

Similar to VISTA, MATSim, the Multi-Agent Transport Simulation toolkit,
stands as another large-scale modeling system that has proven effective
in conducting large-scale incident simulations. Despite certain
similarities, MATSim presents attributes that could potentially offer a
more accurate representation of real-world scenarios and driver
behaviors. Contrary to the model crafted by Pal and Sinha, MATSim
operates as an open-source framework, making it ideal for adapting to
different scenarios. Notably, MATSim facilitates the integration of
real-world data, enhancing the authenticity and precision of network
simulations. This capability is particularly pertinent to this project,
as prior research conducted by (\textbf{hyer2023?}) has made recent UDOT
incident and IMT data available, which could be integrated into the
simulation model for more robust and realistic results. Although MATSim
has been employed to assess network impacts of incidents, its
application in incident management studies remains a largely unexplored
area, presenting an opportunity for further investigation and potential
breakthroughs in the field.

Kaddoura and Nagel conducted a comprehensive study on incident modeling
using MATSim, representing transport users as individual agents within
an iterative framework that allows for adjustments to travel plans both
within and between iterations (Kaddoura \& Nagel, 2018). They accessed
incident data, including Traffic Message Channel (TMC) information
detailing cause and severity, via the HERE application programming
interface. This rich dataset enabled the classification of incidents as
either long-term, such as multiple-day lane closures, or short-term,
affecting transport supply for less than a day. Applying their model to
an inner-city network in Berlin, Germany, Figure~\ref{fig-berlin_cap}
illustrates the modeled incident severities, including a crash on the
southern inner-city motorway ring road that resulted in a full road
closure and several construction sites causing partial capacity
reductions.

\begin{figure}

{\centering \includegraphics{figures/berlin_capacity.png}

}

\caption{\label{fig-berlin_cap}Traffic incidents and their capacity
reduction mapped on the Berlin network.}

\end{figure}

Kaddoura \& Nagel (2018) found that long-term traffic incidents increase
traffic congestion and the average car travel time by 313 sec (+18\%)
per trip. Short-term traffic incidents increase the average travel time
per car trip by another 136 sec (+8\%). Additionally, they found that
for 44\% of all car trips, the agent's transport route contained at
least one road segment for which the capacity or speed limit was reduced
because of an incident. Their study concluded that networks in which
transport users had high levels of knowledge about the incidents and
resulting traffic congestion still experienced an increase in travel
time caused by long and short-term incidents. Finally, Kaddoura and
Nagel asserted that ``accounting for traffic incidents makes the model
more realistic, allowing for an improved policy investigation''
(Kaddoura \& Nagel, 2018, p. 885). The modeling performed by Kaddoura
and Nagel is just one example of research on MATSim's capacity for
incident-based simulations.

A MATSim model developed by Li \& Ferguson (2020) included various
rescheduling options, such as departure time, mode choice, and trip
cancellation. Their simulation found that if travelers received notice
of an incident, they would either depart early from their place of
origin or switch to public transport (Li \& Ferguson, 2020). The process
proposed by Li and Ferguson is beneficial because it allows agents to
reassess their mode choice or route assignment based on the notice of a
reported incident. Li and Ferguson show that users care about total
travel time and travel time variability (risk tolerance to a certain
degree). The receiving of notifications about incidents by agents
impacted both factors. They concluded that ``the provision of real-time
traffic information is a useful approach to mitigating the side-effects
of incidents through helping transport users efficiently adapt their day
plans'' (Li \& Ferguson, 2020, p. 96). Additionally, they found that
``most of the travelers notified of being affected by incidents are
simulated to depart early or switch to public transport, which
effectively reduces the average travel time delay caused by
disruptions'' (Li \& Ferguson, 2020, p. 96). Their findings validate the
conclusions of Sisiopiku et al. (2007) that making incident information
available to agents leads to decreases in travel time and congestion.

This subsection highlight the capabilities of DTA models to simulate the
complexities of traffic incidents, congestion, and travel times. It
highlights how incorporating incident management responses into the
models would enhance their ability to simulate realistic traffic
conditions and support policy analysis. Given its capacity to model
large-scale networks, integrate real-world data, and replicate realistic
driver behavior, MATSim is deemed particularly suitable for this
project. It will be employed to assess the effectiveness of IMT in Utah.

\hypertarget{summary}{%
\section{Summary}\label{summary}}

This literature review provides an overview of the extensive research on
the effectiveness of IMT in reducing RCT and ECU during incident
responses. It has also highlighted studies focused on optimizing the
size and distribution of IMT. However, these studies do not include the
broader implications of incidents and IMT responses on large-scale
networks and their agents. On the other hand, while DTA modeling studies
have effectively explored the impact of incidents on network dynamics
and driver behavior, they often fall short in examining the influence of
IMT or other incident management strategies. This creates a research gap
in understanding the effectiveness of IMT and the impact of incidents on
congested networks. Bridging this gap is crucial, as it enables
researchers to better grasp how alterations in incident occurrence or
IMT availability might influence overall traffic conditions. In our
research, we aim to merge these two areas of study, attempting to model
incident response within a simulation framework, thereby broadening the
scope for evaluating IMT deployment strategies and their effectiveness.

\bookmarksetup{startatroot}

\hypertarget{sec-methods}{%
\chapter{Methodology}\label{sec-methods}}

As highlighted in the \protect\hyperlink{sec-literature}{Literature
Review}, there is substantial evidence indicating that IMT can
effectively reduce RCT and ECU following traffic incidents.
Additionally, the effectiveness of DTA models in analyzing the impact of
such incidents has been well-documented. However, there is a lack of
comprehensive research evaluating IMT impact on entire traffic networks
and their associated agents. To address this gap, it is necessary that
we develop a model capable of simulating both traffic incidents and the
ensuing IMT interventions, with the objective of gauging the efficiency
of IMT deployments. Due to its proficiency in regional-scale incident
simulation and its authentic portrayal of driver behavior, MATSim has
been identified as the most suitable model for this research. This
section describes the methodology, expounding on the model's
capabilities, the requisite data inputs, and the benchmarks established
for determining IMT efficacy.

Our methodology is structured around three main components: the
functionality of the MATSim model, the setup of IMT vehicles and
incidents, and the scenarios for comparative analysis. In the
\protect\hyperlink{sec-MATSim_mod}{Model Design in MATSim} we describe
the structure of the model and the functions it uses to represent
incidents and IMT response. In \protect\hyperlink{sec-model_imp}{Model
Implementaiton} we outline the structure of the simulation by first
describing \protect\hyperlink{sec-IMT_setup}{IMT Setup}, then
\protect\hyperlink{sec-inc_data}{Incident Data and Sampling} and
conclude by describing the \protect\hyperlink{sec-scenarios}{Scenarios}
used for evaluating the impact of incidents and IMT.

\hypertarget{sec-MATSim_mod}{%
\section{Model Design in MATSim}\label{sec-MATSim_mod}}

MATSim is an open-source framework used for conducting extensive,
agent-based transportation simulations on a large scale. Operating as a
dynamic traffic simulation, it plays crucial roles in demand modeling
and agent-based mobility analysis (Dobler \& Nagel, 2016). Thanks to its
open-source architecture, MATSim enables the seamless integration of a
diverse array of modules and packages into its models. Users across the
platform can create, import, and modify these components, fostering a
collaborative and innovative environment.

For the purposes of our research, we developed the ImtModule, a
specialized MATSim extension designed to process incidents and IMT
responses within the simulation. This module leverages existing research
on incident simulation, Demand Rapid Transit (DRT), event handling, and
vehicle dispatch algorithms, building upon these foundations to enhance
the functionality of our model.

In this section, we describe some of the specific tools within MATSim
that we used and adapted to construct a comprehensive and functional
model for both processing and analysis. These tools include Network
Change Events, Vehicle Assignment, and Incident Response. Together, they
contribute to the realism and precision of our traffic simulations,
particularly in the context of responding to roadway incidents, ensuring
that our model provides accurate and reliable results.

\hypertarget{sec-MATSim_Score}{%
\subsection{Replanning and Scoring}\label{sec-MATSim_Score}}

\textless\textless{} I need to read up in the MATSim textbook and look
at our model a little bit more to write this section well. I'll get to
it once I'm done editing the rest of this section
\textgreater\textgreater{}

\hypertarget{sec-NCE}{%
\subsection{Network Change Events}\label{sec-NCE}}

Within a MATSim network, each link is characterized by specific
attributes such as type, length, number of lanes, free-flow speed, and
capacity. To effectively simulate unexpected events and their subsequent
impacts on traffic flow, it is essential to dynamically adjust these
attributes. This capability, termed a Time-Dependent Network, is
elaborated upon in the MATSim textbook (Rieser, 2016) and is vital for
ensuring the realism and accuracy of our simulation.

Network Change Events (NCE) serve as the mechanism within MATSim for
modifying network attributes at precise moments during a simulation.
Detailed in Section 6.1 of the MATSim textbook (Rieser, 2016), the
implementation of NCE requires specific adjustments to the MATSim
configuration file to facilitate a time-variant network. These events
can modify a link's free-flow speed, number of lanes, or capacity. To
initiate a network change event, the system requires specific
information including the time of the event \texttt{startTime}, the
affected link(s) \texttt{link\ refID}, the type of change
\texttt{free-flow\ speed}, \texttt{lanes}, or \texttt{capacity}, and the
value of the change. NCE are the tools used in this study to demonstrate
the impact of both incidents and IMT arrivals.

\hypertarget{sec-imt_response}{%
\subsection{IMT Assignment and Response}\label{sec-imt_response}}

Within MATSim, the dispatch of one or more IMT is prompted by the
occurrence of an incident. The optimal IMT for the situation is
determined using a least-cost path dispatch algorithm, which bases its
calculations on vehicle paths while considering factors such as
congestion and link speed. The success of these methods heavily relies
on the IMT units' capability to navigate through traffic. In cases where
all IMT units are occupied at the time of an incident, the algorithm
waits until a unit becomes available, and subsequently dispatches it to
the incident site.

Upon an IMT's arrival at an incident site, a NCE is activated via an
event handler, a MATSim tool that functions to log simulation events in
real-time. While incidents reduce a link's capacity, the IMT's arrival
triggers a NCE that restores 25\% of the capacity gap on the affected
link. The capacity gap is defined as the difference between the link's
full capacity and its reduced capacity during an incident. In situations
requiring the response of multiple IMT, each arriving unit restores an
additional 25\% of the existing capacity gap.

Figure~\ref{fig-imt_capacity_restore} demonstrates the potential impact
of an incident lacking IMT intervention, compared with scenarios that
include the response of one or two IMT units. This illustration
highlights the critical role of IMTs, showcasing their ability to
mitigate incident impacts on network traffic flow.

\begin{figure}

{\centering \includegraphics{03_methods_files/figure-pdf/fig-imt_capacity_restore-1.pdf}

}

\caption{\label{fig-imt_capacity_restore}IMT capacity restoration upon
arrival example.}

\end{figure}

\hypertarget{sec-model_imp}{%
\section{Model Implementaiton}\label{sec-model_imp}}

To run the MATSim model with the ImtModule extension, a number of input
resources are necessary. For the model to function properly, the
following are needed:

\begin{itemize}
\tightlist
\item
  A plans file detailing the agents to be modeled, as well as their
  origins and destinations.
\item
  A network file with interconnected links, enabling travel for the
  agents specified in the plans file.
\item
  A configuration file outlining the scoring metrics of the simulation
  and establishing parameters pertaining to agent and IMT travel
  patterns.
\item
  An IMT file outlining the IMT starting locations and hours of
  operation.
\item
  An incidents file containing the necessary data to randomly effect NCE
  throughout the simulation.
\end{itemize}

The network and plans files used were developed and calibrated by Lant
(2021) and Day (2022) as part of their research projects studying
accessibility and ride-hailing throughout the Wasatch Front. The
configuration file used in the model was adapted from the Kaddoura \&
Nagel (2018) file, which was used for their MATSim incident analysis
study. It was slightly altered to accommodate the IMT development, but
the parameters they set were largely left unaltered. The IMT file was
produced using data provided by UDOT and UHP, as outline in
\protect\hyperlink{sec-MATSim_mod}{IMT Setup}. The incident data was
produced by (\textbf{hyer2023?}) in his research of IMT performance
measures and is explain in \protect\hyperlink{sec-inc_data}{Incident
Data and Sampling}.

\hypertarget{sec-IMT_setup}{%
\subsection{IMT Setup}\label{sec-IMT_setup}}

UDOT currently operates a fleet of 20 IMT, distributed across three
zones corresponding to Davis, Salt Lake, and Utah counties within the
Wasatch Front. Figure~\ref{fig-IMT_Map} provides a visual representation
of the county boundaries and the initial locations of both existing and
newly proposed IMT vehicles used in simulated scenarios.

\begin{figure}

{\centering \includegraphics{figures/imt_gray_map.png}

}

\caption{\label{fig-IMT_Map}IMT starting locations example map.}

\end{figure}

In Figure~\ref{fig-IMT_Map}, circles denote existing IMT vehicles, and
stars represent proposed additions. These starting locations are
strategically chosen to ensure a balanced distribution of IMT across
each county. The specific locations are recorded in the IMT file as
starting links, facilitating their integration into MATSim. Although
Figure~\ref{fig-IMT_Map} illustrates the distribution of IMT during the
evening shift, it's important to note that the allocation of IMT remains
consistent across the morning and afternoon shifts. Even with a
30-minute overlap between shifts, placing two IMT on the same link does
not cause any operational issues. This setup, while not mirroring the
real-world practice of IMT drivers starting their shifts from home,
serves as a practical approach for our simulations. Given that a
substantial portion of IMT operations occur along major interstates such
as I-15, I-80, and I-215, and considering the daily variations in
starting locations, this positioning along key routes is well-justified.
In the 30-IMT scenario, all vehicles from the 20-IMT scenarios are
included, supplemented by an additional 10 IMT, ensuring coverage across
all three counties.

Our investigation primarily focuses on analyzing the potential impacts
of increasing the IMT fleet, rather than on the influence of their
starting locations. We hypothesize that an increase in the number of
IMT, assuming uniform distribution, could enhance their effectiveness.

Additionally, vehicle scheduling is a critical aspect of our setup. All
three counties operate IMT vehicles across day and afternoon shifts,
with specific timeframes detailed in the truck file loaded into MATSim.
Although IMT vehicles typically do not cross county borders during
operations, our MATSim network does not impose such constraints,
allowing for vehicle movement across counties based on incident
proximity.

\hypertarget{sec-inc_data}{%
\subsection{Incident Data and Sampling}\label{sec-inc_data}}

(\textbf{hyer2023?}) undertook concurrent IMT research and compiled a
comprehensive dataset of incidents requiring IMT intervention, drawing
on data from the UHP. He carefully ensured the completeness of each
incident record in the dataset, capturing crucial details such as the
incident start and end times, RCT, location, and extent of capacity
reduction.

Analyzing data from 2018 and 2022, (\textbf{hyer2023?}) successfully
identified 411 unique incidents with varying degrees of severity,
ranging from property damage to fatal incidents. We utilized this
carefully curated dataset to selectively include specific incidents in
the MATSim model. It is crucial to acknowledge that these 411 incidents
only constitute a portion of all incidents reported by UHP during this
time frame. A significant number of additional incidents were not
considered in the analysis due to the absence of vital metrics necessary
for a comprehensive evaluation (e.g., start time, RCT, etc.).
Nevertheless, the integration of the 411 analyzed incidents with the
additional incomplete records provides insights, aiding in the
quantification of the total number of incidents within a specific time
period. These combined incident records were used in modeling daily
incident frequencies.

To generate ten distinct values representing daily incident frequencies,
we employed a randomized sampling technique. These values were
collectively termed Current Incident Frequencies as they were derived
from the original distribution of daily incidents. Furthermore, we
formulated a second set of ten daily incident values, named Increased
Incident Frequencies. These values were extracted from the upper portion
of the 2022 incident data and were specifically designed to assess the
resilience and efficacy of the IMT system under scenarios of markedly
increased daily incidents. The visual depiction of the original
distribution of daily incidents, alongside the distributions for both
the Current and Increased Incident Frequencies, is illustrated in
Figure~\ref{fig-incident_sampling_plot}.

\begin{figure}

{\centering \includegraphics{03_methods_files/figure-pdf/fig-incident_sampling_plot-1.pdf}

}

\caption{\label{fig-incident_sampling_plot}Incident sampling
distributions for current and increased incident frequencies.}

\end{figure}

In total, twenty values were selected, evenly split with ten allocated
to the Current Incident Frequency category, and the remaining ten to the
Increased Incident Frequency category. Each value was subsequently
paired with a unique three-digit seed number, utilized internally within
MATSim to ensure a randomized selection of incidents for each simulation
scenario. Following this, we employed the MATSim
\protect\hyperlink{sec-NCE}{Network Change Events} to integrate the
incidents into the simulation.

\hypertarget{sec-scenarios}{%
\subsection{Scenarios}\label{sec-scenarios}}

In the \protect\hyperlink{sec-inc_data}{Incident Data and Sampling}
section and Figure~\ref{fig-incident_sampling_plot}, we observe the
establishment and categorization of twenty distinct incident seeds into
either Current or Increased Incident frequencies. Each seed gave rise to
three separate simulation groups. The first group, Incident, exclusively
features scenarios with incidents occurring without any intervention
from IMT. In the second group, includes incidents and the deployment of
20 IMT, while the third group features incidents managed with 30 IMT. To
facilitate efficient organization and comparative analysis, each
scenario was assigned a unique identifier, such as ``1-10-257.'' Within
this coding system, the first digit specifies the simulation group
(Incident, 20 IMT, or 30 IMT), the second digit denotes the number of
incidents included in the simulation, and the third digit corresponds to
the seed value utilized for random incident selection.

In total, six scenario groups were established, as follows:

\begin{itemize}
\tightlist
\item
  No IMT, current incident frequency
\item
  No IMT, increased incident frequency
\item
  20 IMT, current incident frequency
\item
  20 IMT, increased incident frequency
\item
  30 IMT, current incident frequency
\item
  30 IMT, increased incident frequency
\end{itemize}

It is important to note, as described in the IMT Setup, that not all IMT
vehicles are operational simultaneously. Due to scheduling constraints,
the actual number of vehicles on the road at any given time is typically
half of the total fleet size.

The study utilizes the MATSim model for conducting simulations across
all three groups: Incidents, 20-IMT, and 30-IMT. Each scenario underwent
an internal comparison, as well as comparison against a Baseline
scenario devoid of incidents or IMT intervention. This comprehensive
analysis affords a holistic insight into the IMT effects on traffic
dynamics and their operational efficiency.

The primary metric for traffic impact analysis in this study is the
total vehicle hours of delay (VHD), dissected through three
investigative tiers: Network Links, Motorway Links, and Impacted Links.
Network Links offer a macroscopic view of the network-wide delay,
Motorway Links focused on major highways and freeways, and Impacted
Links provide a microscopic view of the delays at incident sites and
their immediate upstream links. This tiered approach ensures a thorough
analysis, capturing the overarching impact on traffic flow while also
honing in on critical areas affected by incidents. The comparative
analysis across different groups and tiers highlights the general
impacts of IMT and their efficacy, particularly at the Impact Link
level.

In addition to VHD, the study investigates the performance and
operational efficiency of IMT. Metrics such as average travel times,
distances, and incident response times of IMT are compared across
scenarios. The 20-IMT group reflects the current fleet of IMT, while the
30-IMT group introduces an additional 10 vehicle, providing a basis for
comparison to discern the impact of fleet size on operational
efficiency. The analysis encompasses both the time and distance traveled
by IMT trucks to incidents, offering insights into how effectively these
resources are deployed and utilized.

By comparing the performance metrics of IMT trucks across different
scenarios, we gain a better understanding of how fleet size influences
operational efficiency. This, in turn, informs strategic decisions
regarding resource allocation and deployment, ensuring that IMT vehicles
are optimally utilized to mitigate traffic delays and enhance roadway
safety.

\bookmarksetup{startatroot}

\hypertarget{sec-results}{%
\chapter{Results}\label{sec-results}}

This section details the outcomes of the Utah Incident Management Team
Optimization project, employing the MATSim model to execute a series of
simulations across a range of scenarios. Specifically, the
scenarios---\emph{Incidents}, \emph{20-Truck}, and
\emph{30-Truck}---were compared with a `Baseline' scenario, facilitating
an evaluation of the repercussions of traffic incidents and the
effectiveness of Incident Management Teams (IMTs) in alleviating traffic
disruptions. In total, 61 simulation scenarios were conducted, each
undergoing 450 iterations within the MATSim framework to produce a
convergence towards a state of equilibrium in travel behavior for most
scenarios. This section offers an analysis of the simulation results,
concentrating on pivotal comparative metrics such as vehicle hours of
delay (VHD), the implications of incidents, and the response dynamics of
the IMTs.

\hypertarget{scenario-results-from-matsim-simulations}{%
\section{Scenario Results from MATSim
Simulations}\label{scenario-results-from-matsim-simulations}}

The investigation employed the MATSim model to perform 20 simulations
for each of the three scenarios: \emph{Incidents}, \emph{20-Truck}, and
\emph{30-Truck}. In the \emph{Incidents} scenario, a random subset of
incidents was generated, without any intervention from Incident
Management Teams (IMTs). The \emph{20-Truck} scenario included the same
random incidents, coupled with responsive measures from UDOT's existing
fleet of 20 IMT vehicles. The \emph{30-Truck} scenario mirrored the
\emph{20-Truck} framework, albeit with an increased fleet, incorporating
an additional 10 IMT vehicles.

These scenarios underwent comparisons against each other as well as
against a \emph{Baseline} scenario with no incidents or IMT deployment.
In total, the study executed 61 simulation scenarios, each completing a
total of 450 iterations within the MATSim model. This iterative approach
ensured that a majority of scenarios gravitated towards a state where
travel plans displayed minimal variance between successive iterations,
effectively achieving an equilibrium in travel behavior by the
simulation's end.

The following sections of this chapter are structured to a analysis of
the results derived from these scenarios. This analysis uses comparative
metrics such as vehicle hours of delay (VHD), the consequences of
traffic incidents, and the dynamics of the IMT responses in relation to
these incidents.

\hypertarget{vehicle-hours-of-delay}{%
\section{Vehicle Hours of Delay}\label{vehicle-hours-of-delay}}

The primary metric used in analyzing the simulation scenarios is the
total vehicle hours of delay (VHD). This analysis is structured across
three distinct tiers: \emph{Network Links}, \emph{Motorway Links}, and
\emph{Impacted Links}.

Overall, it is apparent that the IMT vehicles' influence within the
simulations is most apparent at the ``Impact Link'' level. However,
their effects permeate every tier of analysis. While the outcomes of
certain scenarios align with anticipated projections, others deviate
from intuitive or hypothesized results. The subsequent analysis will
commence with an exploration of the \emph{Network Links} comparisons.

\hypertarget{network-hours-of-delay}{%
\subsection{Network Hours of Delay}\label{network-hours-of-delay}}

The scenarios can be categorized based on incident response (i.e.,
\emph{Incidents}, \emph{20-Truck}, and \emph{30-Truck}), the frequency
of incidents (i.e., \emph{Current Frequency} and \emph{Increased
Frequency}), or a combination of both. In evaluating the results of the
60 scenarios across all network links, or \emph{Network Links}, groups
were categorized by both incident response and frequency. The
Table~\ref{tbl-network_delays_table} provides a summary of the average
total vehicle hours of delay (VHD) for each scenario groupings,
categorized by incident response and frequency of incidents.

The scenarios can be categorized based on incident response (i.e.,
\emph{Incidents}, \emph{20-Truck}, and \emph{30-Truck}), the frequency
of incidents (i.e., \emph{Current Frequency} and \emph{Increased
Frequency}), or a combination of both. In evaluating

\hypertarget{tbl-network_delays_table}{}
\begin{table}
\caption{\label{tbl-network_delays_table}Average Delay for Scenario Groups }\tabularnewline

\centering
\begin{tabular}[t]{llrr}
\toprule
\textbf{Group} & \textbf{Incident Frequency} & \textbf{Average VHD} & \textbf{Change (\%)}\\
\midrule
\cellcolor{gray!6}{Baseline} & \cellcolor{gray!6}{-} & \cellcolor{gray!6}{74568} & \cellcolor{gray!6}{0.0}\\
Incidents & Current & 103159 & 38.3\\
\cellcolor{gray!6}{Incidents} & \cellcolor{gray!6}{Increased} & \cellcolor{gray!6}{104178} & \cellcolor{gray!6}{39.7}\\
20 IMT & Current & 96697 & 29.7\\
\cellcolor{gray!6}{20 IMT} & \cellcolor{gray!6}{Increased} & \cellcolor{gray!6}{95678} & \cellcolor{gray!6}{28.3}\\
\addlinespace
30 IMT & Current & 93769 & 25.7\\
\cellcolor{gray!6}{30 IMT} & \cellcolor{gray!6}{Increased} & \cellcolor{gray!6}{93560} & \cellcolor{gray!6}{25.5}\\
\bottomrule
\end{tabular}
\end{table}

\textless\textless{} I am going to fix the labeling of the tables an
plots so they are easier to understand and follow the scenario groupings
we established in the text -D.J. \textgreater\textgreater{}

Comparing the results of each scenario groupings, its observed that
scenarios including an IMT fleet of 30 vehicles experience the lowest
average VHD. Following closely were the scenarios including the current
fleet (20 IMTs), and, as anticipated, the incident-only scenarios
registered the maximal Total VHD values.

Contrary to the anticipated correlation between more incidents and
longer delays, the VHD patterns and incident frequency relationship is
complex. In incident-only scenarios, a mere one percent increase in
average total VHD occurs with increased incidents, relative to current
frequency scenarios. Interestingly, scenarios incorporating both
existing and increased IMT fleets exhibit a slight reduction in average
delay, despite an uptick in incident frequency.

\textless\textless{} TO DO: try to explain why it is that there is so
little difference between the current and increased incident frequency
groupings. Can it be attributed to the incidents that were selected? Is
that something that you can discover when looking into the impacted
links VHD? \textgreater\textgreater{}

Given that the table reveals average delay values aggregated across
multiple scenarios within each grouping, a graphical representation can
enhance clarity regarding the inherent variance within these data
clusters. Figure~\ref{fig-network_violin_plot} illustrates this data
through various violin plots.

\begin{figure}

{\centering \includegraphics{04_results_files/figure-pdf/fig-network_violin_plot-1.pdf}

}

\caption{\label{fig-network_violin_plot}Average network delay violins.}

\end{figure}

Within this visualization, each violin illustrates the density
distribution of delay values. The increased width in certain sections of
the violin signifies regions where a number of simulations converged
around a specific delay value. Conversely, the narrow sections denote
fewer simulations converging around that delay metric. A dashed
horizontal line has been superimposed to further explain each scenario,
signifying the Baseline scenario as a reference point for comparison.
Additionally, the diamond markers situated within each plot symbolize
the mean delay for all simulations in the respective scenario group.

Upon inspection of these violins, several observations can be made. For
instance, in the \emph{Incidents} scenario, the delay distribution is
characterized by a relatively constricted spread at the lower end,
fanning out around the 100,000 mark, with the mean delay skewing closer
to 104,000, influenced by the values approaching 120,000 total hours of
delay. The distribution within the \emph{20-Truck} scenario exhibits
significant variability, reflected by its consistent width. The scenario
with the \emph{30-Truck} fleet produces marginally reduced variability
in comparison to the \emph{20-Truck} scenario, and contains pronounced
sections around 90,000 VHD and 100,000 VHD.

While the preceding section discussed delays for the entire network,
it's relevant to highlight that all simulated incidents occurred on
motorway links, chiefly along the major interstates of Utah's Wasatch
front. This includes key routes such as I-15, I-80, I-215, among other
prominent freeways and highways. The following section delves into the
simulation outcomes specifically pertaining to these motorway links.

\hypertarget{motorway-link-hours-of-delay}{%
\subsection{Motorway Link Hours of
Delay}\label{motorway-link-hours-of-delay}}

In MATSim, the term ``motorway'' is used to describe a type of link,
which may be referred to as ``freeway'' or ``highway'' in different
contexts. For the sake of consistency, this document primarily uses
``motorway links'' to refer to these road segments. Still, note that in
the simulation, incidents occur on both Interstates and major highways
along the Wasatch Front.

For the comparison of average simulation performances, the
Table~\ref{tbl-motorway_delays_table} table provides a detailed
breakdown of the average Total Motorway VHD for each grouping,
categorized by scenario and incident frequency values.

\hypertarget{tbl-motorway_delays_table}{}
\begin{table}
\caption{\label{tbl-motorway_delays_table}Average VHD of Motorway Links }\tabularnewline

\centering
\begin{tabular}[t]{llrr}
\toprule
\textbf{Group} & \textbf{Incident Frequency} & \textbf{Average VHD} & \textbf{VHD Change (\%)}\\
\midrule
\cellcolor{gray!6}{Baseline} & \cellcolor{gray!6}{-} & \cellcolor{gray!6}{15335} & \cellcolor{gray!6}{0.0}\\
Incidents & Current & 24242 & 58.1\\
\cellcolor{gray!6}{Incidents} & \cellcolor{gray!6}{Increased} & \cellcolor{gray!6}{22321} & \cellcolor{gray!6}{45.6}\\
20 IMT & Current & 18924 & 23.4\\
\cellcolor{gray!6}{20 IMT} & \cellcolor{gray!6}{Increased} & \cellcolor{gray!6}{19176} & \cellcolor{gray!6}{25.0}\\
\addlinespace
30 IMT & Current & 17569 & 14.6\\
\cellcolor{gray!6}{30 IMT} & \cellcolor{gray!6}{Increased} & \cellcolor{gray!6}{18327} & \cellcolor{gray!6}{19.5}\\
\bottomrule
\end{tabular}
\end{table}

Compared to the Table~\ref{tbl-network_delays_table} network\_delays
table, focusing solely on motorway links reveals that the average VHD
surge for incidents-only scenarios relative to the baseline of the same
links exceeds 45\%. Interestingly, in comparison with the full network
delay scenarios, both the \emph{20-IMT} and \emph{30-IMT} scenarios
demonstrated improved average performances on motorways, converging
towards, and occasionally even dropping below, the baseline delay
measures on those links.

Interestingly, the trend observed among the full network links (which
presented a challenge in correlating additional incidents with increased
delays) inverts in the context of motorway links. Here, the IMT
scenarios seem to fare better (indicating reduced delays) in contexts
with fewer incidents, and slightly worse in scenarios with increased
incident counts. However, it's curious to observe that in the
``Incidents'' scenario, the mean delay across scenarios was notably
poorer in simulations with twelve or fewer incidents than in those with
a heightened incident tally.

For a more in=depth understanding of the variances within to the
scenario groupings, the motorway\_violin plot provides additional
insights.

\begin{figure}

{\centering \includegraphics{04_results_files/figure-pdf/fig-motorway_violin_plot-1.pdf}

}

\caption{\label{fig-motorway_violin_plot}Average motorway delay
violins.}

\end{figure}

\textless\textless{} TO DO: fix the labeling on this figure. It's kind
of gnarly \textgreater\textgreater{}

As depicted in the violin plot, both the ``incidents-only'' scenario
groupings exhibit wide spans throughout their plots, signifying a
pronounced variance in delays across the simulations. Notably, the plots
for both the \emph{20-IMT} and \emph{30-IMT} scenarios are wide near the
baseline mean value, which hovers around 15,000 VHD. This expansion,
however, diminishes as the scenarios transition to higher total delay
hours. Particularly in outlier scenarios, the \emph{30-IMT} fleet, equip
with additional vehicles, seems to mitigate delay increases along the
motorway links more effectively than is occasionally evident in the
\emph{20-IMT} vehicle scenarios.

Continuing the discussion, the final tier for comparing VHD is the
\emph{Impacted Links} on the network. These refer to the links where the
simulated incidents take place, as well as the two links immediately
preceding each incident link.

\hypertarget{impacted-links}{%
\subsection{Impacted Links}\label{impacted-links}}

Impacted links are described as the links where an incident occurs,
along with its first two \emph{Feeder} links. These are the two links
through which traffic most commonly flows before reaching the incident
motorway link. Given the variation in link lengths within the motorway,
in certain instances, taking into account just two additional links may
not adequately capture the delay prompted by a specific incident.
Nevertheless, the Table~\ref{tbl-impacted_links} offers significant
insights about how delay on impacted links fluctuates based on the
simulation scenario. For context regarding the table, Total VHD is
computed by considering the duration of the incidents (from start to
finish) and adding one hour post resolution of the incident. In the
simulations conducted, IMT vehicles only enhance the capacity of the
incident link, without shortening its duration. Therefore, for the
`Incident Impact Time', which encompasses the incident duration and an
added hour, the delay values of the incident link and its two Feeder
Links are aggregated to compute a `Delay from Incident {[}hours{]}'
variable, showcased in the subsequent scatter plot. To develop the
summary\_table, the values from all `Delay from Incident {[}hours{]}'
entries within a scenario and incident frequency group were summed. This
sum equates to the Total VHD of delay value for each respective
scenario. The results are presented below:

\hypertarget{tbl-impacted_links}{}
\begin{table}
\caption{\label{tbl-impacted_links}Impacted Links Table }\tabularnewline

\centering
\begin{tabular}[t]{llrr}
\toprule
\textbf{Group} & \textbf{Incident Frequency} & \textbf{Total VHD} & \textbf{Avg. Delay Per Inc. [hrs.]}\\
\midrule
\cellcolor{gray!6}{Baseline} & \cellcolor{gray!6}{Current} & \cellcolor{gray!6}{326} & \cellcolor{gray!6}{3.6}\\
Incidents & Current & 3808 & 42.3\\
\cellcolor{gray!6}{20 IMT} & \cellcolor{gray!6}{Current} & \cellcolor{gray!6}{723} & \cellcolor{gray!6}{8.0}\\
30 IMT & Current & 366 & 4.1\\
\cellcolor{gray!6}{Baseline} & \cellcolor{gray!6}{Increased} & \cellcolor{gray!6}{540} & \cellcolor{gray!6}{2.8}\\
\addlinespace
Incidents & Increased & 3154 & 16.3\\
\cellcolor{gray!6}{20 IMT} & \cellcolor{gray!6}{Increased} & \cellcolor{gray!6}{1645} & \cellcolor{gray!6}{8.5}\\
30 IMT & Increased & 1115 & 5.8\\
\bottomrule
\multicolumn{4}{l}{\textsuperscript{a} Note: Current contains 90 incidents, Increased contains 193}\\
\end{tabular}
\end{table}

The table presents a division of the Baseline scenario into two
segments, for a more accurate comparison with the other three scenarios.
Despite the absence of incidents in the Baseline scenario, its values
are derived from the aggregate delay times of the same links included in
the other scenarios.

From the cumulative delay data, it's clear that, in line with other VHD
comparisons, the VHD in the \emph{30-IMT} scenario most closely mirrors
the baseline. This is followed by the \emph{20-IMT} scenarios and then
the \emph{Incidents-only} groupings. As expected, a rise in incident
frequency leads to an increase in Total VHD. However, the consideration
of additional links makes a direct comparison of these values less
straightforward.

For another perspective, the \emph{Average Delay Per Incident
{[}hours{]}} column divides the total VHD by the number of incidents
within a given category (90 incidents in the current scenario and 193 in
the increased frequency scenario). This calculation reveals something
noteworthy: when comparing the average delay per incident between the
current and increased incident frequencies under the incidents scenario,
there's a significant 160\% increase in the Average Delay Per Incident.
This suggests that the incidents selected in the current scenario might
have been, on average, more impactful than those in the increased
scenario.

For a more detailed observation, the
Figure~\ref{fig-impacted_links_plot} scatter plot below organizes data
by seed type. Incident numbers and seed values serve as y-axis labels
(e.g., ``12\_141'' indicates a scenario with 12 incidents associated
with the seed value 141). The scatter plot is presented as follows:

\begin{figure}

{\centering \includegraphics{04_results_files/figure-pdf/fig-impacted_links_plot-1.pdf}

}

\caption{\label{fig-impacted_links_plot}Delay on impacted links sorted
by seed.}

\end{figure}

The scatter plot provides deeper insight into the nuances of the average
incident impact calculations. In a manner similar to the
Table~\ref{tbl-impacted_links}, the scatter plot presents delay in terms
of average delay per incident, which averages the delays from the
incident links and their corresponding feeder links. At first glance,
the data points representing the Baseline and Increased-IMT scenarios
(in pink and green) seem to situate to the left of the blue dots
depicting the Incidents-only scenarios. However, upon closer
examination, one might discern that in certain scenarios, the
Current-Fleet, Improved-Fleet, and occasionally both, perform
sub-optimally compared to the Incidents-Only scenarios. This might
initially appear counter intuitive, hinting at the possibility that
factors beyond link capacity can influence delay. There exists an
inherent randomness in MATSim's iterative process. The manner in which
agents re-plan their journeys might, at times, influence delay as
significantly, if not more, than variations in link capacity stemming
from incidents and IMT arrivals. Notably, for the four Scenario IDs with
the highest Average Incident Delays, the IMT units appear to
substantially reduce the average delay on the affected links.

Moving ahead, another crucial variable emerges: the performance metrics
associated with the IMT trucks. Such metrics encompass total travel
time, the average distance traversed by each truck, and their average
response time. Similar to VHD, these truck-related metrics might hold
significance for UDOT and other transportation agencies. They will be
elaborated upon in the subsequent section.

\hypertarget{imt-vehicle-analysis}{%
\section{IMT Vehicle Analysis}\label{imt-vehicle-analysis}}

Equally critical to understanding how IMT implementation affects agent
delay and travel time is assessing the efficiency of IMTs in reaching
their intended destinations. This results segment delves into truck
travel behavior, capturing metrics such as average travel times and
distances, along with their typical incident response times. The
analysis encompasses both \emph{20-IMT} fleets scenarios and the
\emph{30-IMT} fleet scenarios.

\hypertarget{imt-travel-time}{%
\subsection{IMT Travel Time}\label{imt-travel-time}}

Travel times for IMT trucks can be extracted from the event files, which
are produced as a standard MATSim output. These files provide insights
into the distance and time traveled by each IMT vehicle. Utilizing this
truck travel data, plots were generated to illustrate the average travel
times and distances for each dispatched truck within a given scenario.
Figure~\ref{fig-truck_time_plot} illustrates the average travel time for
each dispatched vehicle:

\begin{figure}

{\centering \includegraphics{04_results_files/figure-pdf/fig-truck_time_plot-1.pdf}

}

\caption{\label{fig-truck_time_plot}Average truck travel time sorted by
seed.}

\end{figure}

To computed the average travel time per truck we took the cumulative
travel time for all dispatched vehicles in every scenario and then
divided it by the number of trucks deployed. Furthermore, an analysis of
the (\textbf{truck\_csv?}) data reveals that scenarios with an
``Increased'' fleet of 30 vehicles generally dispatched more trucks than
scenarios with a fleet of only 20 vehicles. A nearly analogous
methodology was employed to generate the
Figure~\ref{fig-truck_distance_plot} discussed in the subsequent
section.

\hypertarget{imt-travel-distance}{%
\subsection{IMT Travel Distance}\label{imt-travel-distance}}

As with the variation in truck\_time across different scenario, there is
a clear variance between \emph{20-IMT} and \emph{30-IMT} fleet scenarios
in terms of the average distance traveled per dispatched vehicle. These
results are visualized in the Figure~\ref{fig-truck_distance_plot}.

\begin{figure}

{\centering \includegraphics{04_results_files/figure-pdf/fig-truck_distance_plot-1.pdf}

}

\caption{\label{fig-truck_distance_plot}Average truck distance traveled
sorted by seed.}

\end{figure}

The data presented in the above figure establishes a direct correlation
between an increased number of vehicles and a decrease in average
distance traveled per truck. Notably, the \emph{30-IMT} fleet benefited
in both time and distance scenarios by commencing from identical
locations as the \emph{20-IMT} fleet scenarios. This advantage was made
more noticeable by the addition of vehicles to bridge the spatial
intervals between the existing vehicles.

Altering the starting locations of the vehicles or having them circulate
in a route, as opposed to commencing their activities from a stationary
location, could potentially influence both the time and distance
traveled in either a favorable or adverse manner. However, it's
pertinent to note that the Utah Department of Transportation was not
principally focused on determining the ``optimal'' starting location for
IMT vehicles in this research. While this aspect was not explored in the
present study, it presents an intriguing avenue for future
investigations using the IMT deployment MATSim package associated with
this report.

As highlighted by the work of Lou et al. (2011), in addition to fleet
size, factors such as, vehicle starting locations, and whether IMTs
operate as `roaming entities' can all influence their response times.
Altering the initial positions of the vehicles or directing them along
specific routes, as opposed to a stationary start, can have substantial
impacts on both the time and distance they cover, with potential
advantages or drawbacks. However, it is important to note that
determining the \emph{optimal} starting points for IMT vehicles was not
the primary focus of the Utah Department of Transportation in this
research. While this particular dimension was not explored in our
present study, it provides a compelling direction for future
investigations using the IMT deployment MATSim package associated with
this report.

\hypertarget{imt-response-times}{%
\subsection{IMT Response Times}\label{imt-response-times}}

In the study conducted by Schultz et al. (2019), it is highlighted that,
in Utah, a one-minute increase in IMT response time (RT) results in an
approximate 0.8-minute increase in average roadway clearance time (RCT).
This finding emphasizes the critical factor of IMT response times in
reducing clearance times and subsequent delays. In the simulation
conducted, incidents requested the support of one to four IMT units,
with response times varying across different scenarios.

On average, across 280 simulated incidents, the arrival times in the
\emph{30-IMT} scenarios were 4 minutes faster than those in the
\emph{20-IMT} scenario, as detailed in
Table~\ref{tbl-truck_arrival_table}, which compares the average arrival
times of the 1st, 2nd, 3rd, and 4th trucks.

\hypertarget{tbl-truck_arrival_table}{}
\begin{table}
\caption{\label{tbl-truck_arrival_table}Average Truck Arrival Times }\tabularnewline

\centering
\begin{tabular}[t]{lllllr}
\toprule
\textbf{Group} & \textbf{All Trucks [mins]} & \textbf{1st [mins]} & \textbf{2nd [mins]} & \textbf{3rd [mins]} & \textbf{Total [hours]}\\
\midrule
\cellcolor{gray!6}{20 IMTs} & \cellcolor{gray!6}{15.0} & \cellcolor{gray!6}{11.1} & \cellcolor{gray!6}{21.1} & \cellcolor{gray!6}{28.9} & \cellcolor{gray!6}{105}\\
30 IMTs & 11.0 & 8.9 & 13.2 & 21.2 & 77\\
\cellcolor{gray65}{\cellcolor{gray!6}{\# of Incidents}} & \cellcolor{gray65}{\cellcolor{gray!6}{280}} & \cellcolor{gray65}{\cellcolor{gray!6}{280}} & \cellcolor{gray65}{\cellcolor{gray!6}{116}} & \cellcolor{gray65}{\cellcolor{gray!6}{23}} & \cellcolor{gray65}{\cellcolor{gray!6}{280}}\\
\bottomrule
\end{tabular}
\end{table}

Examining Table~\ref{tbl-truck_arrival_table} provides insight into the
response patterns of IMT vehicles, highlighting their response to 280
out of 283 total incidents across both scenarios, with the remaining 3
incidents falling outside of the IMT operational hours. Within these
responses, 116 incidents requested the assistance of a second IMT unit,
23 required a third, and 3 incidents called for a fourth truck. The
scenarios including 30 IMT units consistently demonstrated faster
average arrival times across nearly all categories, with the exception
of the 4th truck arrival. This particular category, based on a limited
sample of 3 incidents, showed mixed results; the 30-IMT fleet was faster
in two instances but significantly slower in the third. Cumulatively,
the \emph{20-IMT} scenarios resulted in 105 hours of truck travel time,
whereas the \emph{30-IMT} scenarios reduced this total to 77 hours.

Further analysis of (\textbf{truck\_arrival\_data?}) indicates that the
\emph{20-IMT} fleet often dispatched the same truck to multiple
incidents in quick succession, negatively impacting its arrival times.
In contrast, the \emph{30-IMT} scenarios, with more vehicles available,
were less likely to engage in this practice.

\textless\textless{} To do: Consider exploring the dispatch patterns of
the \emph{NEW} trucks in the \emph{30-IMT} scenarios, or alternatively,
analyze the average number of trips per truck, drawing from the previous
discussion. Determine which approach would yield more meaningful
results. \textgreater\textgreater{}

Figure~\ref{fig-truck_arrival_plot} provides a detailed view of how
arrival times varied across scenarios, with data points representing the
difference in arrival times between the \emph{20-IMT} and \emph{30-IMT}
scenarios. Positive values indicate quicker arrival times for the
\emph{30-IMT} fleet, while negative values indicate the opposite. The
general trend suggests that the \emph{30-IMT} fleet consistently
achieved quicker arrival times, particularly in scenarios with a higher
incident count.

\begin{figure}

{\centering \includegraphics{04_results_files/figure-pdf/fig-truck_arrival_plot-1.pdf}

}

\caption{\label{fig-truck_arrival_plot}Difference in IMT Arrival Times,
20 IMTs minus 30 IMTs.}

\end{figure}

From Figure~\ref{fig-truck_arrival_plot}, it becomes evident that the
\emph{30-IMT} vehicle fleet generally boasts shorter arrival times
compared to the \emph{20-IMT} fleet. While outliers exist within each
truck's grouping, it is particularly noticeable that in scenarios with a
higher incidence of incidents (18, 19, 20, and 21), the addition of 10
vehicles significantly enhances the overall arrival times of trucks,
with the most pronounced improvements observed in the arrival times of
the 2nd and 3rd trucks.

In summary, the combined results from the Vehicle Hours of Delay (VHD)
and IMT Vehicle analysis demonstrate the efficacy of IMT vehicles in
mitigating delays, particularly on the road segments directly impacted
by incidents and their immediate surroundings. Additionally, an increase
in the size of the vehicle fleet is directly associated with reductions
in both the average travel time per truck and the response times per
incident, underscoring the benefits of a larger IMT fleet in emergency
response situations.

\bookmarksetup{startatroot}

\hypertarget{sec-conclusions}{%
\chapter{Conclusions}\label{sec-conclusions}}

This section need not be overly long. You should address any limitations
of your results, such as dependence on underlying assumptions or
geographic scope. You should also provide a map for future research.

Finally, you should underline the contributions of this work and any
practical relevance.

\hypertarget{within-day-replanning}{%
\subsection{Within-Day Replanning}\label{within-day-replanning}}

\textless\textless{} The model was meant to incorporates the concept of
within-day replanning to a certain extent, as elaborated in Chapter 30
of the MATSim textbook. I loaded the within-day replanning module but
didn't specify which agents needed to use it or the times that they
needed to use it. \textgreater\textgreater{}

\textless\textless{} Dr.~Macfarlane. Within-day replanning is not being
implemented in the way that I thought that it was. After re-reading the
literature review I realized that I had confused the replanning that
occurs from iteration to iteration with the replanning mentioned by
Kaddoura \& Nagel (2018) that only occurs for specific agents within the
simulation. \textgreater\textgreater{}

\textless\textless{} I do think it is still worth mentioning their
research, and I can discuss how we could have used within-day replanning
more effectively in the limitations section of this report. I am sorry
for the confusion and for not realizing the problem earlier. Perhaps
this section in the Methodology could discuss the strategy the model
uses from iteration to iteration. \textgreater\textgreater{}

\textless\textless{} I'll likely move this section to the limitations
and talk about it, but wanted to leave you a note here just in case you
were looking for this section - Daniel Jarvis\textgreater\textgreater{}

\bookmarksetup{startatroot}

\hypertarget{references}{%
\chapter*{References}\label{references}}
\addcontentsline{toc}{chapter}{References}

\markboth{References}{References}

\hypertarget{refs}{}
\begin{CSLReferences}{1}{0}
\leavevmode\vadjust pre{\hypertarget{ref-bennett2021}{}}%
Bennett, L. S. (2021). \emph{Analysis of benefits of an expansion to
UDOT's incident management program}.

\leavevmode\vadjust pre{\hypertarget{ref-boyles2018}{}}%
Boyles, S. (2018). \emph{Introduction to dynamic traffic assignment}.

\leavevmode\vadjust pre{\hypertarget{ref-day2022}{}}%
Day, C. S. (2022). \emph{Forecasting ride-hailing across multiple model
frameworks}.

\leavevmode\vadjust pre{\hypertarget{ref-dia2006}{}}%
Dia, H., \& Cottman, N. (2006). Evaluation of arterial incident
management impacts using traffic simulation. \emph{Intelligent Transport
Systems, IEE Proceedings}, \emph{153}, 242--252.
\url{https://doi.org/10.1049/ip-its:20055005}

\leavevmode\vadjust pre{\hypertarget{ref-dobler2016}{}}%
Dobler, C., \& Nagel, K. (2016). \emph{Within-day replanning}. {Ubiquity
Press}.

\leavevmode\vadjust pre{\hypertarget{ref-kaddoura2018}{}}%
Kaddoura, I., \& Nagel, K. (2018). Using real-world traffic incident
data in transport modeling. \emph{Procedia Computer Science},
\emph{130}, 880--885. \url{https://doi.org/10.1016/j.procs.2018.04.084}

\leavevmode\vadjust pre{\hypertarget{ref-kim2012}{}}%
Kim, W., Franz, M., Chang, G.-L., \& University of Maryland (College
Park, Md. ). Dept. of C. and E. E. (2012). \emph{Enhancement of freeway
incident traffic management and resulting benefits.}

\leavevmode\vadjust pre{\hypertarget{ref-lant2021}{}}%
Lant, N. J. (2021). \emph{Estimation and simulation of daily activity
patterns for individuals using wheelchairs}.

\leavevmode\vadjust pre{\hypertarget{ref-li2020}{}}%
Li, J., \& Ferguson, N. (2020). A multi-dimensional rescheduling model
in disrupted transport network using rule-based decision making.
\emph{Procedia Computer Science}, \emph{170}, 90--97.
\url{https://doi.org/10.1016/j.procs.2020.03.012}

\leavevmode\vadjust pre{\hypertarget{ref-lou2011}{}}%
Lou, Y., Yin, Y., \& Lawphongpanich, S. (2011). Freeway service patrol
deployment planning for incident management and congestion mitigation.
\emph{Transportation Research Part C: Emerging Technologies},
\emph{19}(2), 283--295. \url{https://doi.org/10.1016/j.trc.2010.05.014}

\leavevmode\vadjust pre{\hypertarget{ref-ozbay2013}{}}%
Ozbay, K., Iyigun, C., Baykal-Gursoy, M., \& Xiao, W. (2013).
Probabilistic programming models for traffic incident management
operations planning. \emph{Annals of Operations Research},
\emph{203}(1), 389--406. \url{https://doi.org/10.1007/s10479-012-1174-6}

\leavevmode\vadjust pre{\hypertarget{ref-pal2002}{}}%
Pal, R., \& Sinha, K. C. (2002). {SIMULATION MODEL FOR EVALUATING AND
IMPROVING EFFECTIVENESS OF FREEWAY SERVICE PATROL PROGRAMS}.
\emph{Journal of Transportation Engineering}, \emph{128}(4).

\leavevmode\vadjust pre{\hypertarget{ref-rieser2016}{}}%
Rieser, H., Nagel. (2016). \emph{MATSim data containers}. {Ubiquity
Press}.

\leavevmode\vadjust pre{\hypertarget{ref-schultz2019}{}}%
Schultz, G. G., Saito, M., Eggett, D. L., Bennett, L. S., Hadfield, M.
G., Civil, B. Y. University. Dept. of, \& Environmental Engineering.
(2019). \emph{Analysis of performance measures of traffic incident
management in utah}.

\leavevmode\vadjust pre{\hypertarget{ref-sisiopiku2007}{}}%
Sisiopiku, V. P., Li, X., Mouskos, K. C., Kamga, C., Barrett, C., \&
Abro, A. M. (2007). Dynamic traffic assignment modeling for incident
management. \emph{Transportation Research Record}, \emph{1994}(1),
110--116. \url{https://doi.org/10.3141/1994-15}

\leavevmode\vadjust pre{\hypertarget{ref-skabardonis1998}{}}%
Skabardonis, A., Petty, K., Varaiya, P., \& Bertini, R. (1998).
Evaluation of the freeway service patrol ({FSP}) in los angeles.
\emph{PATH Research Report}.

\leavevmode\vadjust pre{\hypertarget{ref-wirtz2005}{}}%
Wirtz, J. J., Schofer, J. L., \& Schulz, D. F. (2005). Using simulation
to test traffic incident management strategies: {The} benefits of
preplanning. \emph{Transportation Research Record}, \emph{1923}(1),
82--90. \url{https://doi.org/10.1177/0361198105192300109}

\end{CSLReferences}

\cleardoublepage
\phantomsection
\addcontentsline{toc}{part}{Appendices}
\appendix

\hypertarget{event-handlers}{%
\chapter{Event Handlers}\label{event-handlers}}

This is the event handler

\begin{Shaded}
\begin{Highlighting}[]
\KeywordTok{public} \KeywordTok{class}\NormalTok{ Bike }\OperatorTok{\{}
    \BuiltInTok{Integer}\NormalTok{ gears }\OperatorTok{=} \DecValTok{0}\OperatorTok{;}
    \BuiltInTok{String}\NormalTok{ color }\OperatorTok{=} \StringTok{"red"}\OperatorTok{;}
    \BuiltInTok{Double}\NormalTok{ price }\OperatorTok{=} \FloatTok{500.0}\OperatorTok{;}

    \FunctionTok{Bike} \OperatorTok{(}\BuiltInTok{Integer}\NormalTok{ gears}\OperatorTok{,} \BuiltInTok{String}\NormalTok{ color}\OperatorTok{,} \BuiltInTok{Double}\NormalTok{ price}\OperatorTok{)} \OperatorTok{\{}
        \KeywordTok{this}\OperatorTok{.}\FunctionTok{gears} \OperatorTok{=}\NormalTok{ gears}\OperatorTok{;}
        \KeywordTok{this}\OperatorTok{.}\FunctionTok{color} \OperatorTok{=}\NormalTok{ color}\OperatorTok{;}
        \KeywordTok{this}\OperatorTok{.}\FunctionTok{price} \OperatorTok{=}\NormalTok{ price}\OperatorTok{;}
    \OperatorTok{\}}
\OperatorTok{\}}
\end{Highlighting}
\end{Shaded}


\end{document}
